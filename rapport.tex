\documentclass[a4paper,11pt]{article}
\usepackage[utf8]{inputenc}
\usepackage[francais]{babel}
\usepackage{graphicx}
\usepackage{geometry}
\usepackage{numprint}
\usepackage[squaren,Gray,cdot]{SIunits}
\usepackage{amsmath}
\usepackage{listings}
\usepackage{courier}
\geometry{hmargin=3.5cm, vmargin=2.5cm}

\lstset{language=C, numbers=left}

\usepackage{color}

\definecolor{mygreen}{rgb}{0,0.6,0}
\definecolor{mygray}{rgb}{0.5,0.5,0.5}
\definecolor{mymauve}{rgb}{0.58,0,0.82}

\lstset{ %
	backgroundcolor=\color{white},   % choose the background color; you must add \usepackage{color} or \usepackage{xcolor}; should come as last argument
	basicstyle=\footnotesize,        % the size of the fonts that are used for the code
	breakatwhitespace=false,         % sets if automatic breaks should only happen at whitespace
	breaklines=true,                 % sets automatic line breaking
	captionpos=b,                    % sets the caption-position to bottom
	commentstyle=\color{mygreen},    % comment style
	deletekeywords={...},            % if you want to delete keywords from the given language
	escapeinside={\%*}{*)},          % if you want to add LaTeX within your code
	extendedchars=true,              % lets you use non-ASCII characters; for 8-bits encodings only, does not work with UTF-8
	frame=left,	                   % adds a frame around the code
	keepspaces=true,                 % keeps spaces in text, useful for keeping indentation of code (possibly needs columns=flexible)
	keywordstyle=\color{blue},       % keyword style
	language=C,                 % the language of the code
	morekeywords={*,...},           % if you want to add more keywords to the set
	numbers=left,                    % where to put the line-numbers; possible values are (none, left, right)
	numbersep=5pt,                   % how far the line-numbers are from the code
	numberstyle=\tiny\color{mygray}, % the style that is used for the line-numbers
	rulecolor=\color{black},         % if not set, the frame-color may be changed on line-breaks within not-black text (e.g. comments (green here))
	showspaces=false,                % show spaces everywhere adding particular underscores; it overrides 'showstringspaces'
	showstringspaces=false,          % underline spaces within strings only
	showtabs=false,                  % show tabs within strings adding particular underscores
	stepnumber=1,                    % the step between two line-numbers. If it's 1, each line will be numbered
	stringstyle=\color{mymauve},     % string literal style
	tabsize=4,	                   % sets default tabsize to 2 spaces
	title=\lstname                   % show the filename of files included with \lstinputlisting; also try caption instead of title
}



\begin{document}
	
	
	
	
	\section{Voir le monde en 7 couleurs}
	
	
	\subsection{Remplissage aléatoire du monde}
	
	On crée la fonction fill\_rand dans laquelle on va remplir le board avec des couleurs aléatoires, ainsi que les territoires de départ de chacun des joueurs dans les coins.
	
	On rapelle que la fonction random génère une suite de nombres (des \texttt{int}) générés à partir d'une graine. Cette suite est unique pour chaque graine. Il est donc impropre de parler de génération de nombres aléatoires, mais plutot de génération \emph{pseudo-aléatoire}. On affecte à la graine la valeur du temps UNIX au moment du d\'emarrage du programme.
	
	On affecte ensuite à chaque élement de la liste \texttt{board} une valeur comprise entre \texttt{'A'} et \texttt{'G'}. Comme \texttt{rand} renvoie un \texttt{int} compris entre 0 et une constante \texttt{RAND\_MAX} supérieure à 32767, on fait un modulo 7 pour avoir seulement des entiers compris entre 0 et 6.
	
	On affecte ensuite les valeurs au coins en haut à droite et en bas à gauche aux deux joueurs.
	
	\begin{lstlisting}
/** Fills the borad with random colors */
void fill_rand(void)
{
	int i;
	for(i = 0 ; i < BOARD_SIZE*BOARD_SIZE ; i++)
	{
		// Setting a random value between 'A' and 'G'
		board[i] = (rand() % 7) + 'A';
	}
	
	// Setting the start cells for each player in the corners
	set_cell(BOARD_SIZE-1, 0, '^');
	set_cell(0, BOARD_SIZE-1, 'v');
}\end{lstlisting}
	
	
	\subsection{Mise à jour du board après un coup}
	\label{MAJ}
	
	Selon les règles, chaque joueur choisit à tour de rôle une couleur ; il prend le contrôle de toutes les cases adjacentes à son territoire qui sont de la couleur choisie. Si une case de la couleur choisie est adjacente à une case capturée, elle l'est aussi automatiquement.
	
	On va écrire une fonction \texttt{void updateBoard(char color, char player)}, qui prend pour paramètres le caractère correspondant à la couleur choisie par le joueur, et la caractère associé au joueur (dans notre cas \texttt{v} ou \texttt{\^}).
	
	La stratégie va être la suivante : on vérifie pour chaque case de la couleur choisie si ses voisines appartiennent au joueur. Si c'est le cas, cette case est capturée. On répète le processus, jusqu'à ce qu'aucun changement ne puisse être fait.
	
	Pour ce faire, une utilisera un flag tel qu'on boucle tant que celui-ci est \texttt{true}. On lui affecte la valeur \texttt{false} à chaque début d'itération ; si une modification est faite, on lui affecte la valeur \texttt{true} ; si aucune modification n'est faite, le flag restera avec la valeur \texttt{false}, et la boucle se terminera.
	
	On fera attention \`a ne pas contr\^oler des cases qui sont en-dehors du plateau : en particulier, on ne devra pas contr\^oler la case \`a droite d'une case qui est d\'ej\`a sur le rebord droit du plateau. Ainsi, on v\'erifie d'abord que la case en cours n'est pas sur le rebord droit avant de contr\^oler la case \`a sa droite. M\^eme chose pour les autres rebords.
	
	\begin{lstlisting}
void updateBoard(char* boardToUpdate, char color, char player_c)
{
	// Defining those not to compute them at each iteration ...
	int BB = BOARD_SIZE * BOARD_SIZE;
	int BBMO = BOARD_SIZE * (BOARD_SIZE-1);
	
	/* While flag=1 the check continues
	* if flag=0 at the end of the loop, the check ends
	*/
	int flag = 1;
	while(flag)
	{
		flag = 0;
		
		int i;
		for(i = 0 ; i < BB ; i++)
		{
			/* If the block is not at the top of the board, checking if the upper block belongs to player
			* '' left '' left ''
			* '' bottom '' lower ''
			* '' right '' right ''
			* the order in the && blocks prevent any segfault
			*/
			if(boardToUpdate[i] == color &&
			( (i >= BOARD_SIZE && boardToUpdate[i-BOARD_SIZE] == player_c) ||
			((i % BOARD_SIZE) && boardToUpdate[i-1] == player_c) ||
			((i < BBMO) && boardToUpdate[i+BOARD_SIZE] == player_c) ||
			((i % BOARD_SIZE != BOARD_SIZE - 1) && boardToUpdate[i+1] == player_c) ) )
			{
				boardToUpdate[i] = player_c;
				
				// If a change is made, continue checking
				flag = 1;
			}
		}
		
		// If no change is made, flag remains equal to 0 and the loop ends
	}
}\end{lstlisting}
	
	On peut vérifier que l'algorithme fonctionne simplement en faisant une partie joueur contre joueur et en vérifiant les cas limites. Pour cela, on a besoin de répondre aux questions 4 et 5.
	
	
	
	\section{À la conquête du monde}
	
	L'objectif est ici d'implémenter le mode de jeu humain contre humain. Chaque joueur jouera un coup à tour de rôle, jusqu'à ce que le territoire de l'un des joueurs atteigne ou dépasse les 50\% du board. On implémentera ceci dans une fonction \texttt{void gamehvh()} (pour game human v. human).
	
	
	\subsection{Jeu infini (et au-delà)}
	
	On va en premier lieu récupérer un input : chaque joueur devra écrire le caractère correspondant à la couleur qu'il veut capturer. Pour cela nous allons utiliser la fonction \texttt{scanf}, récupérer le caractère entré dans un \texttt{char}, puis le passer en paramètre dans la fonction \texttt{updateBoard} définie en partie \ref{MAJ}.
	
	Pour pouvoir faire jouer les deux joueurs à tour de rôle, on introduit la variable \texttt{int player} qui prend la valeur 0 quand c'est au joueur 1 de jouer et la valeur 1 quand c'est au joueur 2 de jouer. Il suffit d'inverser sa valeur à la fin de chaque itération.
	
	On boucle tout cela à l'infini dans un premier temps.
	
	\begin{lstlisting}
void gamehvh()
{
	int player = 1;
	char input_color;
	
	// Infinite loop ...
	while(1)
	{
		if(player) // Player 1
		{
			// Getting Player1's entry
			printf("Player 1 : ");
			scanf(" %c", &input_color);
			// Updating the board according to it
			updateBoard(board, toupper(input_color), 'v');
		}
		else // Player 2
		{
			// Ditto ...
			printf("Player 2 : ");
			scanf(" %c", &input_color);
			updateBoard(board, toupper(input_color), '^');
		}
		
		printf("\n");
		print_board();
		
		// The other player's turn
		player = !player;
	}
}
	\end{lstlisting}
	
	On pensera \`a mettre un espace devant \texttt{\%c} : lors des premiers tests (sans l'espace), la saisie du joueur 2 \'etait outrepass\'ee, car le programme extrayait sucessivement le caract\`ere correspondant \`a la couleur choisie pour le joueur 1, puis le caract\`ere \guillemotleft retour \`a la ligne \guillemotright pour le joueur 2.
	
	De plus, on peut ajouter \texttt{toupper} pour \'eviter d'avoir \`a \'ecrire une majuscule \`a chaque fois.
	
	
	\subsection{Toutes les bonnes choses ont une fin}
	
	Il s'agit maintenant de gérer la condition de fin de partie. D\`es qu'un joueur atteint 50\% du terrain, on d\'eclenche un return revoyant 1 si le joueur 1 gagne, et 2 si le joueur 2 gagne. On utilisera la fonction \texttt{rate} qui renvoie la proportion de terrain occup\'ee par un caract\`ere donn\'e dans un board donn\'e. On en profitera pour l'afficher pour chacun des joueurs.
	
	\begin{lstlisting}
float rate(char* boardToCheck, char character)
{
	int count = 0;
	int BB = BOARD_SIZE*BOARD_SIZE;
	
	int i;
	for(i = 0; i < BB; i++)
	{
		// Increment the counter at each encounter
		if(boardToCheck[i] == character)
		count ++;
	}
	
	return (float)count/BB;
}\end{lstlisting}
	
	Remarque : on aurait pu garder la boucle infinie \texttt{while(true)} et utiliser le mot-clé \texttt{break} pour sortir de la boucle immédiatement \dots
	
	\begin{lstlisting}
int gamehvh()
{
	int player = 1;
	char input_color;
	
	/* The game continues until one player wins (return)
	*/
	while(1)
	{
		if(player) // Player 1
		{
			// Getting Player1's entry
			printf("Player 1 : ");
			scanf(" %c", &input_color);
			// Updating the board according to it
			updateBoard(board, toupper(input_color), 'v');
		}
		else // Player 2
		{
			// Ditto ...
			printf("Player 2 : ");
			scanf(" %c", &input_color);
			updateBoard(board, toupper(input_color), '^');
		}
		
		printf("\n");
		print_board();
		
		// Getting the fraction of the board belonged by each player
		float rate1 = rate(board, 'v');
		float rate2 = rate(board, '^');
		printf("Player 1 : %f %% \t\tPlayer 2 : %f %% \n\n", 100*rate1, 100*rate2);
		
		// If a player gets more than half the board
		if(rate1 >= 0.5f)
		{
			// The game finishes and 1 wins
			return 1;
		}
		else if(rate2 >= 0.5f)
		{
			// The game finishes and 2 wins
			return 2;
		}
		
		// The other player's turn
		player = !player;
	}
}\end{lstlisting}
	
	
	
	\section{La statégie (??!) de l'aléa}
	
	
	Dans un premier temps, codons une fonction g\'en\'erique pour jouer une partie contre un bot, quelqu'il soit. Cela \'evitera de faire des copier-coller, et permetrra d'avoir le code de l'IA encapsul\'e dans une fonction qui renvoie simplement le coup \`a jouer (i.e. la couleur choisie par l'IA).
	Cette fonction, que l'on apellera \texttt{gamehvb} (pour game human v. bot) prendra comme param\`etre un pointeur vers la fonction de l'IA contre laquelle on souhaite jouer.
	
	\begin{lstlisting}
void gamehvb(char (*AIFunc)(char))
{	
	char input_color;
	
	/* The game continues until one player wins (return)
	*/
	while(1)
	{
		// Player's turn
		printf("Player : ");
		scanf(" %c", &input_color);
		updateBoard(board, toupper(input_color), 'v');
		
		// Bot's turn, according to the given AI function
		updateBoard(board, (*AIFunc)('^'), '^');
		
		printf("\n");
		print_board();
		
		// Getting the fraction of the board belonged by each player
		float rate1 = rate(board, 'v');
		float rate2 = rate(board, '^');
		printf("Player : %f %% \t\tAI : %f %% \n\n", 100*rate1, 100*rate2);
		
		// If a player gets more than half the board
		if(rate1 >= 0.5f)
		{
			// The game finishes and the human wins
			return 1;
		}
		else if(rate2 >= 0.5f)
		{
			// The game finishes and the bot wins
			return 2;
		}
	}
}\end{lstlisting}
	
Ainsi, si l'on veut jouer contre l'IA \texttt{whateverAI}, on codera une fonction \texttt{char whateverAI(char botChar)} renvoyant la couleur \`a jouer en fonction de l'\'etat du board. L'argument botChar est le caract\`ere correspondant \`a l'IA (soit \texttt{\^} soit \texttt{v}). On appellera la fonction \texttt{gamehvb} ainsi :
	
	\begin{lstlisting}
gamehvb(&whateverAI);\end{lstlisting}
	
	
	\subsection{L'IA al\'eatoire}
	
	Pour cette IA, on choisit n'importe quelle couleur au hasard (Waouh !).
	
	\begin{lstlisting}
char dumbRandBotAI(char botChar)
{
	// Returning a random character between 'A' and 'G'
	return 'A' + rand()%7;
}\end{lstlisting}
	
	L'argument \texttt{botChar} n'est ici pas utilis\'e. Cette IA aura le m\^eme comportement quelque soit l'\'equipe \`a laquelle elle appartient (\texttt{\^} ou \texttt{v}).
	
	
	\subsection{L'IA al\'eatoire am\'elior\'ee}
	
	Cette IA devra choisir au hasard parmi les couleurs qui lui permettent d'\'etendre son terrain (m\^eme si son choix n'est pas optimum). Pour cela, on cr\'ee une copie du terrain et on simule un coup pour chaque couleur dans cell-ci. On compare ensuite la proportion de terrain conquise avant et apr\`es la simulation. Si on a gagn\'e du terrain, on stocke la couleur dans un tableau, dans lequel on viendra s\'electionner au hasard l'une d'entre elles.
	
	\begin{lstlisting}
char lessDumbBotAI(char botChar)
{
	int BB = BOARD_SIZE*BOARD_SIZE;
	
	/* Copy of the board
	* used to foresee the future plays
	*/
	char* futureBoard = malloc(BB * sizeof(char));
	
	// Rate before the hypothetic plays
	float prevRate = rate(board, botChar);
	
	float curRate;
	
	int availableColorCnt = 0;
	
	// List of the colors that make the AI gain ground
	char* chosenColors = malloc(7 * sizeof(char));
	
	// Loop for each color
	char c;
	for(c = 'A'; c <= 'G'; c++)
	{
		// Reset the copy
		int i;
		for(i = 0; i < BB; i++)
		futureBoard[i] = board[i];
		
		// Simulate a play with a color
		updateBoard(futureBoard, c, botChar);
		
		// Checking if the territory has expanded
		curRate = rate(futureBoard, botChar);
		if(curRate > prevRate)
		{
			// If so, add the color to the possibly chosen colors
			chosenColors[availableColorCnt] = c;
			availableColorCnt++;
		}
	}
	
	free(futureBoard);
	
	// Randomly choose one of the selected colors
	char chosenColor = chosenColors[rand()%availableColorCnt];
	
	free(chosenColors);
	
	return chosenColor;
}\end{lstlisting}
	
	
	\section{La loi du plus fort}
	
	\subsection{L'IA gloutonne}
	
	Cette IA devra choisir la couleur qui maximisele nombre de cases prises lors du tour. Pour cela, on va faire pareil que pour le bot al\'eatoire am\'elior\'e : on va simuler un coup pour chaque couleur. L'IA choisit ensuite la couleur pour laquelle sa part de terrain est maximale.
	
	\begin{lstlisting}
char greedyBotAI(char botChar)
{	
	int BB = BOARD_SIZE*BOARD_SIZE;
	
	/* Copy of the board
	* used to foresee the future plays and choose the best
	*/
	char* futureBoard = malloc(BB * sizeof(char));
	
	char chosenColor;
	
	float curRate;
	float maxRate;
	
	// Loop for each color
	char c;
	for(c = 'A'; c <= 'G'; c++)
	{
		// Reset the copy
		int i;
		for(i = 0; i < BB; i++)
		futureBoard[i] = board[i];
		
		// Simulate a play with each color
		updateBoard(futureBoard, c, botChar);
		
		// Finding the color with maximum rate
		curRate = rate(futureBoard, botChar);
		if(curRate > maxRate)
		{
			maxRate = curRate;
			chosenColor = c;
		}
	}
	
	return chosenColor;
}\end{lstlisting}
	
	
	\subsection{Le tournoi}
	
	On fait se combatrre les deux IA l'une contre l'autre. Pour cela, on peut faire une fonction g\'en\'erique, qui organise un tournoi entre deux IA.
	
	\begin{lstlisting}
float contestbvb(char (*AIFunc1)(char), char (*AIFunc2)(char))
{
	// Victory count of AI1
	int victoryCnt1 = 0;
	
	int i;
	for(i = 0; i < 100; i++)
	{
		// Reset the board before each round
		fill_rand();
		
		// Fight !
		int winner = gamebvb(AIFunc1, AIFunc2);
		
		if(winner == 1)
		victoryCnt1++;
	}
	
	return (float)victoryCnt1/100;
}\end{lstlisting}
	
	On ne renvoie que la proportion de victoires de l'IA 1, mais celle de l'IA 2 peut se d\'eduire par un calcul savant.
	
	Le combat entre l'IA al\'eatoire et l'IA gloutonne pr\'esente assez peu de suspense : l'IA gloutonne gagne 100\% des matches \`a chaque tournoi.
	
	Pour que le match siot \'equitable, il faudrait que le choix de n'importe quelle couleur donne autant de gain, et ce \`a chaque tour pour n'importe quelle couleur choisie. Ainsi, la stat\'egie al\'eatoire et la stat\'egie gloutonne deviendraient \'equivalentes.
	
	
	
	
	
	
\end{document}
